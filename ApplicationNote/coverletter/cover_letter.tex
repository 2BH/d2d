\documentclass[12pt,a4paper]{scrartcl}
\usepackage[utf8]{inputenc}
\usepackage{amsmath, amssymb}
\usepackage{array}
\usepackage{listings} 
\usepackage{epsfig} 
\usepackage{graphicx} 
\usepackage{rotating} 
\usepackage{lscape}
\usepackage{color}
\usepackage{xcolor} 
\usepackage{subfig}
\usepackage{everysel}
\usepackage{natbib}
\usepackage{times}
\usepackage{bibentry}

% page size
\setlength{\parskip}{0.4\baselineskip}
\setlength{\textheight}{21 cm}
\setlength{\textwidth}{15 cm}
\setlength{\topmargin}{0.5 cm}
\setlength{\hoffset}{-0,5 cm}
\setlength{\voffset}{-0,5 cm}
\setlength{\headsep}{1 cm}
\setlength{\oddsidemargin}{1 cm}
\setlength{\evensidemargin}{1 cm}
\marginparwidth 1.8cm \marginparsep 10pt
\setlength{\belowcaptionskip}{5pt}
\setlength{\belowcaptionskip}{5pt}

% caption style
\renewcommand{\captionfont}{\footnotesize}
\renewcommand{\captionlabelfont}{\bf \sffamily}
\setcapindent{0pt}

% table rulers
\newcommand{\toprule}{\hline\hline}
\newcommand{\midrule}{\hline}
\newcommand{\botrule}{\hline\hline}

% differential opterators
\newcommand{\dd}[2]{\frac{\partial #1}{\partial #2}}
\newcommand{\ddd}[3]{\frac{\partial^2 #1}{\partial #2 \partial #3}}
\newcommand{\DD}[2]{\frac{\mathrm{d} #1}{\mathrm{d} #2}}
\newcommand{\DDsquare}[2]{\frac{\mathrm{d}^2 #1}{\mathrm{d} #2^2}}
\newcommand{\DDD}[3]{\frac{\mathrm{d}^2 #1}{\mathrm{d} #2\mathrm{d} #3}}


% hyperlink color
\usepackage[colorlinks=true, linkcolor=black, citecolor=black, filecolor=black, 
urlcolor=black]{hyperref}
%\usepackage[colorlinks=true, linkcolor=blue, citecolor=blue, filecolor=blue, urlcolor=blue]{hyperref}

% comments
\newcommand{\ar}[1]{{\color{red}#1}}

\begin{document}

\pagestyle{empty}

\quad
\vspace{1cm}

\begin{flushright}
Cambridge, March 12, 2015
\end{flushright}

\noindent Dear Editors,

\noindent In Systems Biology, ordinary differential equation models are often used to analyze signal transduction pathways, gene regulation or cellular decisions. One of the most critical steps in this approach is to construct models based on large sets of data generated under complex experimental conditions and to perform efficient and reliable parameter estimation for model fitting. We present the \textbf{Data2Dynamics} software environment that was tailored to solve these challenges.

\noindent The software is freely available, open source and developed in a community effort using a web-based code hosting service and a revision control system. It has been applied in several projects that led to publications, see below. Some of those applications are provided as benchmark examples within the software for further methods development and as guideline for novel applications. The software was awarded twice as best performer in the Dialogue for Reverse Engineering Assessments and Methods (DREAM, 2011 and 2012). 

\noindent Beyond modeling of biological systems the software can be applied to any comparable problem in other fields of research.

\noindent We would be more than happy if you would consider our \textbf{Application Note} for publication in Bioinformatics.

\noindent Best wishes,\\
Andreas Raue and colleagues
\vspace{1cm}

\bibliographystyle{plainnat_mod}
\nobibliography{./publications_using_d2d}

\subsubsection*{Biology publications making use of the Data2Dynamics software}

\begin{itemize}
\item \bibentry{DAlessandro:2015vl}
\item \bibentry{Klett:2015fp}
\item \bibentry{Verbruggen:2014ve}
\item \bibentry{Muller:2014nr}
\item \bibentry{Kanodia:2014fv}
\item \bibentry{Beer:2014rz}
\item \bibentry{Gin:2013ly}
\item \bibentry{Muller:2013ty}
\item \bibentry{Muller:2013qv}
\item \bibentry{Muller:2013sf}
\item \bibentry{Boehm:2014rm}
\item \bibentry{Bachmann:2011fk}
\item \bibentry{Raia:2011vn}
\item \bibentry{Becker:2010hs}
\end{itemize}

\subsubsection*{Methodology publications making use of the Data2Dynamics software}

\begin{itemize}
\item \bibentry{Meyer:2014kq}
\item \bibentry{Toensing:2014}
\item \bibentry{Raue:2014mz}
\item \bibentry{Kreutz:2013uq}
\item \bibentry{Raue:2013fk}
\item \bibentry{Schelker:2012uq}
\item \bibentry{Hug:2012fk}
\item \bibentry{Vehlow:2012fk}
\item \bibentry{Steiert:2012fk}
\item \bibentry{Raue:2012zt}
\item \bibentry{Kreutz:2011kx}
\item \bibentry{Raue:2010fk}
\item \bibentry{Raue:2010ys}
\item \bibentry{Raue:2009ec}
\end{itemize}


\subsubsection*{Benchmark applications distributed with the Data2Dynamics software}

\begin{itemize}
\item \bibentry{Raia:2011vn}
\item \bibentry{Bachmann:2011fk}
\item \bibentry{Becker:2010hs}
\item \bibentry{Swameye:2003hc}
\end{itemize}

\end{document}
